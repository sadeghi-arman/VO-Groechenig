\documentclass{article}
\usepackage[usenames,dvipsnames]{color,xcolor}
\usepackage{listings}
\usepackage{hyperref}
\usepackage{amsmath,amsthm,amssymb,mathtools,fancyhdr,tcolorbox,cancel,pgfplots,tikz}
\usetikzlibrary{intersections}
\usepackage[right=3.75cm,left=3.75cm,top=4cm,bottom=4cm]{geometry}

\theoremstyle{definition}
\newtheorem{definition}{Definition}
\newtheorem{ub}{Aufgabe}
\newtheorem*{lo*}{Lösung}
\newtheorem*{lem*}{Lemma}

\newcommand{\ggt}[2]{\mathrm{ggT}(#1,#2)}

\begin{document}
	
	\thispagestyle{plain}
	\begin{minipage}{5cm}
		\includegraphics[width=5cm]{logo}\\
		\centering
		Fakultät für Mathematik
	\end{minipage}
	\hfill
	\begin{minipage}{7cm}
		\baselineskip=.4cm
		Arman Sadeghi Rad\\
		Matrikelnummer 12223560 \\
		Einführung in das mathematische Arbeiten \\
		Wintersemester 2022/2023
	\end{minipage}\\[1mm]
	\hrule height2pt \vskip1mm
	\noindent
	Übungsblatt 5B
	\hrule height2pt \vskip1mm

\begin{ub}
	Der gr\"o\ss te gemeinsame Teiler von $ a $ und $ m $ ist 
	$ \ggt{a}{m} $. Daher ist auch die Aussage
	\[ 
	\ggt{\frac{a}{\ggt{a}{m}}}{\frac{m}{\ggt{a}{m}}} = 1 
	 \]
	richtig. Nach einem Satz von der Vorlesung gibt es $ p,q \in \mathbb{Z} $, sodass
	\[ 
	\frac{a}{\ggt{a}{m}}p + \frac{m}{\ggt{a}{m}}q = 1. \tag{1}\label{1}
	 \]
	Nach einer Multiplikation mit $ b $ wird hergeleitet, dass
	\[ 
	abp\frac{1}{\ggt{a}{m}} + mbq\frac{1}{\ggt{a}{m}} = b
	 \]
	und in der Resklasse von $ m $
	\[ 
	abp\frac{1}{\ggt{a}{m}} \equiv b \mod m \tag{2}\label{2}
	 \]
	F\"ur die Gleichungen \ref{1} und \ref{2} in der Restklasse von $ m $, gelten auch $ p + km $ wenn $ k \in \mathbb{Z} $. aber wenn 
	$ k = \ggt{a}{m} $ dann gilt
	\[ 
	ab(p + m\ggt{a}{m})\frac{1}{\ggt{a}{m}} = abp\frac{1}{\ggt{a}{m}} + m \frac{\ggt{a}{m}}{\ggt{a}{m}} 
	\equiv abp\frac{1}{\ggt{a}{m}} \equiv b \mod m. 
	 \]
	Daher gibt es nur $ \ggt{a}{m} $ L\"osungen die mod $ m $ inkongruent sind.
\end{ub}

\end{document}