\documentclass{article}
\usepackage[usenames,dvipsnames]{color,xcolor}
\usepackage{listings}
\usepackage{hyperref}
\usepackage{amsmath,amsthm,amssymb,mathtools,fancyhdr,tcolorbox}
\usepackage[right=3.75cm,left=3.75cm,top=4cm,bottom=4cm]{geometry}

\theoremstyle{definition}
\newtheorem{definition}{Definition}
\newtheorem{ub}{Aufgabe}
\newtheorem*{lo*}{Lösung}

\begin{document}
	
	\thispagestyle{plain}
	\begin{minipage}{5cm}
		\includegraphics[width=5cm]{logo}\\
		\centering
		Fakultät für Mathematik
	\end{minipage}
	\hfill
	\begin{minipage}{7cm}
		\baselineskip=.4cm
		Arman Sadeghi Rad\\
		Matrikelnummer 12223560 \\
		Einführung in das mathematische Arbeiten \\
		Wintersemester 2022/2023
	\end{minipage}\\[1mm]
	\hrule height2pt \vskip1mm
	\noindent
	Übungsblatt 1B
	\hrule height2pt \vskip1mm
\begin{tcolorbox}
	\begin{ub}
		\begin{enumerate}
			\item $ 2k + 2l = 2(k+l) $
			\item $ 2k + 2l + 1 = 2(k+l) + 1 $
		\end{enumerate}
	\end{ub}
	\begin{ub}
		Wenn alle der Zahlen $ 2 $ sind.
	\end{ub}
\end{tcolorbox}

\end{document}