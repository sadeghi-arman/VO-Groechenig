\documentclass{article}
\usepackage[usenames,dvipsnames]{color,xcolor}
\usepackage{listings}
\usepackage{hyperref}
\usepackage{amsmath,amsthm,amssymb,mathtools,fancyhdr,tcolorbox,cancel,pgfplots,tikz,enumerate}
\usetikzlibrary{intersections}
\usepackage[right=3.75cm,left=3.75cm,top=4cm,bottom=4cm]{geometry}


\theoremstyle{definition}
\newtheorem{definition}{Definition}
\newtheorem{ub}{Aufgabe}
\newtheorem*{lo*}{Lösung}
\newtheorem*{lem*}{Lemma}


\newcommand{\ggt}[2]{\mathrm{ggT}(#1,#2)}

\begin{document}
	
	\thispagestyle{plain}
	\begin{minipage}{5cm}
		\includegraphics[width=5cm]{logo}\\
		\centering
		Fakultät für Mathematik
	\end{minipage}
	\hfill
	\begin{minipage}{7cm}
		\baselineskip=.4cm
		Arman Sadeghi Rad\\
		Matrikelnummer 12223560 \\
		Einführung in das mathematische Arbeiten \\
		Wintersemester 2022/2023
	\end{minipage}\\[1mm]
	\hrule height2pt \vskip1mm
	\noindent
	Übungsblatt 6B
	\hrule height2pt \vskip1mm

\begin{ub}
\begin{enumerate}[(i)]
	\item 
		601 ist eine Primzahl. Nach dem kleinen Satz von Fermat, $ p = 601 $ und wenn $ a $ und $ p $ teilerfremde Zahlen sind, $ a^{p-1} \equiv 1 \mod p $. Dann gilt
	\[ 
	27^{4800} \equiv \left( 27^{600} \right)^8 \equiv 1^8 \equiv 1  \mod p.
	\]
	\item
	\[ 
	27^{300} \equiv 3^{900} \equiv 3^{600} \cdot 3^{300} \equiv 1 \cdot 3^{300} \mod p.
	 \]
	
\end{enumerate}
\end{ub}
\begin{ub}
	\begin{enumerate}[(i)]
		\item 
		Sei $ G $ die Gruppe $ \mathbb{Z}_4 $
		\[ 
		f_2(3 \cdot 3) = f_2(9) = 2 \cdot 3 \cdot 3 = 18 \equiv 2 \not\equiv 
		0 \equiv 36 = (2 \cdot 3)\cdot (2 \cdot 3) = f_2(3) \cdot f_2(3).
		 \]
		und $ f_a $ ist ein Homomorphismus, genau dann wenn $ a^2 = a $. Sei $ a $ ein idempotentes Element, dann $ f_a(xy) = axy = a^2xy = axay = f_a(x)f_a(y) $. Und wenn $ f_a $ ein Homomorphismus ist, dann $ f_a(xy) = f_a(x)f_a(y) $ und $ axy = axay = a^2axy $ und $ a $ ist ein idempotentes Element.  
		\item 
		Die Abbildung ist nach Definition wohldefiniert. Und wenn $ f_a(x) = f_a(y) $ dann $ ax = ay $. Da $ G $ eine Gruppe ist, kann daraus gefolgert werden, dass $ x = y $. Au\ss erdem f\"ur jedes 
		$ f_a $ existiert ein $ a \in G $. 
	\end{enumerate}
\end{ub}
\begin{ub}
	\begin{enumerate}[(i)]
		\item 
		Erstens zeigen wir, dass die Abbildung surjektiv ist. Nehmen wir an, dass $ (a,b) $ ein beliebiges Element von $ \mathbb{Z}_m \times \mathbb{Z}_n $ ist. Um das entsprechende Element von $ \mathbb{Z} _{mn}$
		zu finden, sollte man das System 
		\[ 
		\begin{array}{ll}
			x \equiv a & \mod m \\
			x \equiv b & \mod n
		\end{array}
		\]
		l\"osen. Da $ m $ und $ n $ teilerfremde Elemente sind, nach dem chinesischem Restsatz, gibt es eine 
		L\"osung dazu. Zun\"achst pr\"ufen wir, ob die Abbildung wohldefiniert ist. Wenn $ x = y $ dann 
		$ x \equiv y \mod m $ und $ x \equiv y \mod n $. Daher $ f(x) = f(y) $. Zum Schluss beweisen wir, dass die Abbildung injektiv ist. Wenn $ (x_1, y_1) = (x_2 , y_2) $, dann 
		$ x_1 \equiv x_2 \mod m  $ und $ x_1 \equiv x_2 \mod n $. Da $ (m,n) = 1 $, gilt
		$ x_1 \equiv x_2 \mod mn $. \"Ahnlicherweise k\"onnte man diese Aussage auch f\"ur $ y_1 $ und $ y_2 $ beweisen. 
		\item 
		\[ 
		(x_1,y_1) + (x_2,y_2) := (x_1 + x_2 , y_1 + y_2), \quad 
		(x_1,y_1) \cdot (x_2,y_2) := (x_1x_2,y_1y_2).  
		 \]
	\end{enumerate}
\end{ub}
\begin{ub}
	\begin{enumerate}[(i)]
		\item 
		\[ 
		0^2 \equiv 0, 1^2 \equiv 1, 2^2 \equiv 4, 3^2 \equiv 4, 4^2 \equiv 1 \mod 5.
		 \]
		\[ 
		0^2 \equiv 0, 1^2 \equiv 1, 2^2\equiv 4, 3^2 \equiv 2, 4^2 \equiv 2, 5^2 \equiv 4, 6^2 \equiv 
		1 \mod 7
		 \]
		$ a \in \{ 0,1,4 \} $ f\"ur $ n=5 $ und $ a \in \{ 0,1,2,4 \} $ f\"ur $ n=7 $.
		\item 
		Wenn die Kongruenz eine L\"osung $ d $ hat, dann es gibt genau zwei L\"osungen $ d $ und
		$ -d $. Angenommen es gibt eine andere inkongruente L\"osung wie $ k $. Dann gilt
		\[ 
		k^2 \equiv d^2 \equiv (k-d)(k+d) \mod p.
		 \]
		Da $ k \neq d $ und $ k \neq -d $, $ p $ ist die Produkt von zwei positiven Zahlen. es ist ein Widerspruch zu der Annahme, dass $ p $ eine Primzahl ist.
		\item 
		\[ 
		\begin{array}{r|c|c}
			x & x^2 & x^2 \mod 35 \\ \hline
			1 & 1 & \boxed{1} \\
			2 & 4 & 4 \\
			3 & 9 & 9 \\
			4 & 16 & 16 \\
			5 & 25 & 25 \\
			6 & 36 & \boxed{1} \\
			7 & 49 & 14 \\
			8 & 64 & 29 \\
			9 & 81 & 11 \\
			10 & 100 & 30 \\
			11 & 121 & 16 \\
			12 & 144 & 4 \\
			13 & 169 & 29 \\
			14 & 196 & 21 \\
			15 & 225 & 15 \\
			16 & 256 & 11 \\
			17 & 289 & 9 \\
			-17 & 289 & 9 \\
			-16 & 256 & 11 \\
			-15 & 225 & 15 \\
			-14 & 196 & 21 \\
			-13 & 169 & 29 \\
			-12 & 144 & 4 \\
			-11 & 121 & 16 \\
			-10 & 100 & 30 \\
			-9 & 81 & 11 \\
			-8 & 64 & 29 \\
			-7 & 49 & 14 \\
			-6 & 36 & \boxed{1} \\
			-5 & 25 & 25 \\
			-4 & 16 & 16 \\
			-3 & 9 & 9 \\
			-2 & 4 & 4 \\
			-1 & 1 & \boxed{1}
		\end{array}
		 \]
	\end{enumerate}
\end{ub}
	
\end{document}