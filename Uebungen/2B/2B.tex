\documentclass{article}
\usepackage[usenames,dvipsnames]{color,xcolor}
\usepackage{listings}
\usepackage{hyperref}
\usepackage{amsmath,amsthm,amssymb,mathtools,fancyhdr,tcolorbox,cancel}
\usepackage[right=3.75cm,left=3.75cm,top=4cm,bottom=4cm]{geometry}

\theoremstyle{definition}
\newtheorem{definition}{Definition}
\newtheorem{ub}{Aufgabe}
\newtheorem*{lo*}{Lösung}

\begin{document}
	
	\thispagestyle{plain}
	\begin{minipage}{5cm}
		\includegraphics[width=5cm]{logo}\\
		\centering
		Fakultät für Mathematik
	\end{minipage}
	\hfill
	\begin{minipage}{7cm}
		\baselineskip=.4cm
		Arman Sadeghi Rad\\
		Matrikelnummer 12223560 \\
		Einführung in das mathematische Arbeiten \\
		Wintersemester 2022/2023
	\end{minipage}\\[1mm]
	\hrule height2pt \vskip1mm
	\noindent
	Übungsblatt 2B
	\hrule height2pt \vskip1mm

\begin{ub}
	Angenommen, es g\"abe solche Zahlen. dann $ 7 | 28m + 42n $ aber nat\"urlich, $ 7 $ ist kein Teiler von 
	$ 100 $. Daher ist es ein Widerspruch zu der Annahme, dass die Gleichung wahr ist. 
\end{ub}
\begin{ub}
	Indirekter Beweis. Angenommen, dass $ n $, eine gerade Zahl und $ n^3 $ eine ungerade Zahl ist.
	$ n $ ist gerade. Dann gilt
	\[ 
	n = 2k + 1 \Rightarrow n^3 = 8k^3 + 1 + 6k + 12k^2 = 2(4k^3 + 6k^2 + 3k) + 1
	 \]
	Es ist ein Widerspruch zu der Annahme, dass $ n^3 $ gerade ist.
\end{ub}
\begin{ub}
	Wir behaupten, dass $ a=b $. $ a|b $ ergibt $ \exists k, b =ak $ und $ b|a $ ergibt $ \exists l, a=bl $. Dann gilt $ a = (b)l = (ak)l $. Daher m\"ussen $ k $ und $ l $ beide $ 1 $ sein und $ a = b\cdot 1 $ und $ b = a \cdot 1 $.
\end{ub}
\begin{ub}
	\begin{enumerate}
		\item 
		Gegenbeispiel. $ 2|2 $ und $ 3|3 $ aber $ 4 \not| 9 $.
		\item 
		Gegenbeispiel. $ 2 \cdot 3 | 5 \cdot 6 $ aber $ 2 \not | 5 $.
		\item $ a|c $, dann gilt $ c=ak $. $ b|d $, dann gilt $ d = bl $. Multiplizieren von den linken Seiten und den rechten Seiten ergibt $ cd = klab $, dann gilt $ ab|cd $.
		\item 
		Gegenbeispiel. $ 2 \cdot 2 | 4 \cdot 1 $ aber $ 2 \not | 1 $.
	\end{enumerate}
\end{ub}
\begin{ub}
	\begin{align*}
		\sum\limits_{k=1}^{42} \frac{1}{k} & = 1 + \frac{1}{2} + (\frac{1}{3} + \frac{1}{4})
		+ ( \frac{1}{5} + \frac{1}{6} + \ldots + \frac{1}{8} ) + (\frac{1}{9} + \frac{1}{10} + \ldots +
		\frac{1}{16}) \\
		& + (\frac{1}{17} + \frac{1}{18} + \ldots + \frac{1}{32}) + (\frac{1}{33} + \ldots \frac{1}{42}) \\
		& \geq 1+ \frac{1}{2} + (\frac{1}{4} + \frac{1}{4}) + (\frac{1}{8} + \frac{1}{8} + \ldots + \frac{1}{8}) + (\frac{1}{16} + \frac{1}{16} + \ldots + \frac{1}{16}) \\
		& + (\frac{1}{32} + \frac{1}{32} + \ldots \frac{1}{32}) + (\frac{1}{64} + \ldots + \frac{1}{64}) \\
		& = 1 + \frac{1}{2} + 2 \cdot \frac{1}{4} + 4\cdot\frac{1}{8} + 8\cdot\frac{1}{16} + 16\cdot\frac{1}{32} + 10\cdot\frac{1}{64} \\
		& = 1 + \frac{1}{2} + \frac{1}{2} + \frac{1}{2} + \frac{1}{2} + \frac{1}{2} + 10\cdot\frac{1}{64}
		\geq 3.
	\end{align*}
\end{ub}
\begin{ub}
	\begin{proof}[Induktionsanfang]
		Es gibt eine Primzahlfaktorisierung f\"ur 2. $ 2 = 2 $.
	\end{proof}
	\begin{proof}[Induktionsschritt]
		Nehmen wir an, dass es eine Primzahlfaktorisierung f\"ur alle Zahlen bis $ n $ gibt. Dann beweisen wir, dass es auch eine f\"ur die Zahl $ n $ selbst gibt. Wäre $ n $ eine Primzahl, dann ist die Aussage selbsverst\"andlich wahr. Wenn nicht, dann gibt es eine Faktorisierung wie $ n = ab $ und 
		$ a,b < n $. Zufolge der Behauptung, gibt es eine Primzahlfaktorisierung f\"ur $ a $ und $ b $ wie
		$ a = p_1 \cdots p_k $ und $ b = q_1 \cdots q_l $. Dann gilt 
		\[ 
		n = ab = p_1\cdots p_k \cdot q_1 \cdots q_l.
		 \] 
		Es ist eine Primzahlfaktorisierung f\"ur $ n $.
	\end{proof}
\end{ub}
\begin{ub}
	Angenommen, dass es keine andere Primzahl gibt. Dann entweder ist $ p_1p_2\cdots p_n +1 $ eine Primzahl oder nicht. W\"are sie eine Primzahl, dann gibt es nichts anderes zu beweisen. Aber wenn nicht, dann gibt es eine Primzahl $ p $ die deren Teiler ist. Angenommen das $ p $ eine von 
	$ p_1 , \ldots, p_n $, z.B $ p_i $ ist, dann gilt
	\[ 
	p_i | p_1p_2\cdots p_n +1 \Rightarrow p_i | 1 
	 \] 
	Es ist ein Widerspruch. Daher sollte sich $ p $ von den anderen unterscheiden.  
\end{ub}
\end{document}